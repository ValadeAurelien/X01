\message{ !name(rapport.tex)}\documentclass[11pt]{article}
\usepackage[utf8]{inputenc}

% MATH
\usepackage{amssymb}
\usepackage{amsmath}
\usepackage{amsthm}
\usepackage{mathtools}

\newtheorem{pb}{Problème}
\newtheorem{rmq}{Remarque}
%\newtheorem{cusdef}{Def}
%\newtheorem{custhm}{Thm}
%\newtheorem{cuscor}{Cor}
%\newtheorem{cusrk}{Rk}
\newenvironment{cusenv}[2]
{\begin{samepage}\noindent\textbf{#1} -- (#2) \par}
{\end{samepage} \bigskip}

\newenvironment{cusdef}[1]
{\begin{cusenv}{Def}{#1}}{\end{cusenv}}

\newenvironment{custhm}[1]
{\begin{cusenv}{Thm}{#1}}{\end{cusenv}}

\newenvironment{cuscor}[1]
{\begin{cusenv}{Cor}{#1}}{\end{cusenv}}

\newenvironment{cusprop}[1]
{\begin{cusenv}{Prop}{#1}}{\end{cusenv}}

\newenvironment{cusrk}[1]
{\begin{cusenv}{Rk}{#1}}{\end{cusenv}}	

% MISE EN FORME
\usepackage[left=2cm,right=2cm,top=2cm,bottom=2cm]{geometry}

\usepackage{hyperref}
\usepackage{xcolor}
\hypersetup{
	colorlinks,
	linkcolor={red!50!black},
	citecolor={blue!50!black},
	urlcolor={blue!80!black}
}


% MACROS
\usepackage{xparse}
\newcommand{\smbox}[1]{\mbox{\footnotesize #1}}
\newcommand{\A}{\mathcal{A}}
\newcommand{\Am}{\mathbb{A}}
\newcommand{\N}{\mathbb{N}}
\newcommand{\C}{\mathbb{C}}
\renewcommand{\P}{\mathbb{P}}
\newcommand{\R}{\mathbb{R}}
\newcommand{\K}{\mathbb{K}}
\newcommand{\M}{\mathbb{M}}
\newcommand{\Ah}{A^{hom}}
\newcommand{\uh}{u^{hom}}
\newcommand{\fh}{f^{hom}}


\newcommand{\ie}{\emph{i.e.} }
\newcommand{\ms}{~~~}
\newcommand{\norm}[1]{\left|\left|#1\right|\right|}
\newcommand{\dual}[1]{#1'}
\newcommand{\sca}[2]{\big<#1, #2\big>}
\newcommand{\cont}[1]{\mathcal{C}^{#1}}
\newcommand{\question}[2]{\paragraph{Question #1.}\textit{#2} \\}
\newcommand{\Hd}{H^1_{\#}}



\title{Rapport TP X01 \\ TP3}
\author{Aurélien Valade}
\date{}

\begin{document}

\message{ !name(rapport.tex) !offset(-3) }

\maketitle

\section{Intruction et structure du code}


Le but de ce TP est de résoudre le problème suivant :

\begin{pb}[$\varepsilon-$problème]
  \label{pb:eps}
  Soit $\varepsilon>0$, soient $A^\varepsilon, ~f^\varepsilon$ bien définies, trouver $u^\varepsilon\in H^1_0(\Omega)$ tel que  
  \[
    \begin{cases}
      \nabla \big(A^\varepsilon(x,x/\varepsilon) \nabla u^\varepsilon\big) = f^\varepsilon(x, x/\varepsilon) \quad \mbox{sur}\quad \Omega\\
      u^\varepsilon = 0 \quad \mbox{sur}\quad \partial\Omega.
    \end{cases}
  \]
\end{pb}
\begin{rmq}
  On a $x =(x_1, x_1) \in \R^2$  et $x/\varepsilon y = (y_1, y_2) \in [0,1]^2$. Dans la suite on note $Y=[0,1]^2$ la cellule élémentaire sur laquelle les coefficients de
  $A^\varepsilon$ sont périodiques. Dans les cas ou on ne travaille que sur une cellule on prend $\varepsilon=1$ et on note $A := A^1$. 
\end{rmq}
\begin{pb}
  \label{pb:hom}
  Trouver $u^0 \in H^1_0(\Omega)$ solution du $\varepsilon-$problème \ref{pb:eps} à la limite $\varepsilon\to 0 $.
  \[
    \begin{cases}
      \nabla \big(\Ah(x) \nabla \uh\big) = \fh(x) \quad \mbox{sur}\quad \Omega\\
      \uh= 0 \quad \mbox{sur}\quad \partial\Omega.
    \end{cases}
  \]
\end{pb}


Pour cela nous allons d'abord calculer numériquement les solutions du problème \ref{pb:eps} pour plusieurs valeurs de $\varepsilon>0$. Dans un second temps nous
introduirons les problèmes de cellules que nous résoudrons numériquement, puis nous pourrons enfin calculer la solution au problème \ref{pb:hom}.

La structure du code ainsi que l'arrangement des dossiers ont été un peu modifiés. Tous les \texttt{*.msh} pour les problèmes de Neumann ou Dirichlet se trouvent
dans le dossier \texttt{geoms/} avec un executable \texttt{bash} pour en créer à volonté. Les \texttt{*.msh} pour les problèmes périodiques se trouvent eux dans
le dossier \texttt{geoms\_per/} de même qu'un script pour les générer.

De plus, le corps de la routine principale se trouve maintenant dans \texttt{principal\_dirichlet\_aux.m}, cependant le fichier script est toujours bien
\texttt{principal\_dirichlet.m}.

\section{Solution exacte}

\subsection{Discrétisation}

Le but de cette partie est d'étudier la convergence effective des solutions numériques vers des solutions analytiques connues. Les résultats présentés ici sont
calculés à partir des routines implémentées pour les TP précédents.

Nous allons donc résoudre le problème \ref{pb:eps} en faisant varier $A^\varepsilon$ et $f^\varepsilon$ de paire de sorte à ce que la solution $u$ soit connue et
ne dépende pas de $\varepsilon$. L'ensemble des matrices de diffusion et leur seconds membres associés se trouvent en \autoref{tab:Af}. Dans tous ces cas on
calcul $f^\varepsilon$ de sorte à ce que la solution soit
\[
  u^\varepsilon(x, y) = u(x, y) = \sin(\pi x)\sin(\pi y).
\]

\begin{table}
  \centering
  \begin{tabular}{c|c|p{0.5\textwidth}}
    type & $A^\varepsilon$ & $f^\varepsilon$ \\
    \hline
    (i) & Id & $2\pi^2\sin(\pi y_1 )\sin(\pi y_2)$ \\
    \hline
    (ii) & $ \left(
           \begin{matrix}
             1 & 0 \\
             0 & 2 
           \end{matrix} \right) $   
                           & $3\pi^2\sin(\pi y_1 )\sin(\pi y_2)$ \\
    \hline
    (iii) & $ \left(
            \begin{matrix}
              2+\sin(2\pi y_1) & 0 \\
              0 & 4 
            \end{matrix} \right)$   
                           & {\small $2\pi^2 (-2 \cos(\pi y_1) \cos(2\pi y_1) +
                             \sin(\pi y_1)\sin(2 \pi y_1) + 3 \sin(\pi y_1))\sin(\pi y_2)$} \\
    \hline
    (iv) & $ \left(
           \begin{matrix}
             2+\sin(2\pi y_1) & 0 \\
             0 & 4+\sin(2\pi y_1)
           \end{matrix} \right)$   
                           & {\small $\pi^2 (-4 \cos(\pi y_1) \cos(2 \pi y_1) + 
                             3 \sin(\pi y_1) \sin(2 \pi y_1) + 6 \sin(\pi y_1)) \sin(\pi y_2)$} \\
    \hline
    (v) & $ ( 2+\sin(2\pi y_1)) ( 4+\sin(2\pi y_2) )$ Id   
                           & {\footnotesize $4 \pi^2                                                                     
                             (- (\sin(2 \pi  y_1) + 1)  \sin(\pi y_1)  \cos(\pi y_2)  \cos(2 \pi  y_2) -
                               (\sin(2 \pi  y_2) + 4)  \sin(\pi y_2)  \cos(\pi y_1)  \cos(2 \pi  y_1) +
                             (\sin(2 \pi  y_1) + 1)  (\sin(2 \pi  y_2) + 4)  \sin(\pi y_1)  \sin(\pi y_2))$} \\
  \end{tabular}
  \caption{Matrices de diffusion et seconds membres associés. Dans un soucie de lisibilité les fonctions sont données pour $\varepsilon=1$. }
  \label{tab:Af}
\end{table}

\section{Solution au problème homogénéisé}

\subsection{Les problèmes de cellule}

La difficulté mathématique à laquelle répond la méthode de l'homogénéisation est le passage du problème \ref{pb:eps} au problème dit homogénéisé
\ref{pb:hom}. Il faut notamment calculer les coeffecients limites de $\Ah$ et $\fh$. Plus précisement, dans notre cas, nous travaillons avec $A^\varepsilon$ parfaitement périodique,
on sait que l'on va avoir $\Ah$ constant. Pour garder la solution
\[
  \uh(x, y) = u(x, y) = \sin(\pi x)\sin(\pi y)
\]
il nous suffira de prendre
\[
  \fh = \pi^2 Tr(\Ah) u(x, y).
\]

Les problèmes de cellules sont au nombre de dimensions spatiales du problème, donc deux dans notre cas. Ceux sont posés dans le problème \ref{pb:cell} sous leurs
formes variationnelles.
\begin{pb}
  \label{pb:cell}
  Trouver $u_i \in V$ tels que $\forall i \in \{1,2\}$
  \begin{equation}
    \forall v \in V, \quad \int_Y A \nabla u_i \nabla v = - \int_Y A e_i \nabla v 
  \end{equation}
  avec
  \[
    V = 
    \left\{
      \psi \in \Hd(Y), \qquad \int_Y \psi = 0
    \right\}.
  \]
\end{pb}

Montrons tout d'abord que ces problèmes sont bien posés. On prend l'une des deux équations, et l'on pose $u:=u_i$, de même on pose
\[
  (u,v) = \int_Y A \nabla u \nabla v, \quad
  l(v) = -\int_Y A e_i \nabla v 
\]

On a bien $a(u,v)$ et $l(v)$ (bi)linéaires. Montrons que $a(u,v)$ continue 
\begin{align*}
  \big|a(u,v)\big| &= \left| \int_Y A \nabla u \nabla v \right| \\
                   &\leq \norm{A\nabla u}_{L^2} \norm{\nabla v}_{L^2} && \mbox{Cauchy Schwarz} \\
                   &\leq \beta \norm{\nabla u}_{L^2} \norm{\nabla v}_{L^2} && \mbox{A bornée} \\
                   &\leq \beta \norm{u}_{\Hd} \norm{v}_{\Hd} && \norm{\cdot}_{L^2}<\norm{\cdot}_{\Hd} \\
\end{align*}

Montrons que $l(v)$ est continue
\begin{align*}
  \big|l(v)\big| &= \left| \int_Y A e_i \nabla v \right|  && \text{définition} \\
                 &\leq \norm{Ae_i}_{L^2} \norm{\nabla v}_{L^2} && \text{Cauchy Schwarz} \\
                 &\leq \beta_i \norm{\nabla v}_{L^2} && \text{$A$ bornée, donc $A e_i$ aussi} \\
                 &\leq \eta_i \norm{v}_{\Hd} && \norm{\cdot}_{L^2}<\norm{\cdot}_{\Hd}
\end{align*}

Montrons que $a(u,v)$ est coercive
\begin{align*}
  a(u,u) &= \int_Y A \nabla u  \nabla u \\
         &\geq \xi \int_Y \nabla u^2  && \text{$A$ minorée par $\xi$}\\
         &\geq \xi \norm{\nabla u}^2_{L^2} \\
         &\geq \zeta \norm{u}^2_{\Hd} && \text{Pointcaré dans $V$} 
\end{align*}
avec $\zeta = \frac{\xi}{(C^2+1)}$, $C$ étant la constante de poincaré associée à $V$.

Cependant, les problèmes étant mal conditionnés sous ces formes, on va plutôt résoudre des problèmes dit dégradés 
\begin{pb}[Problèmes dégradés]
  \label{pb:celleta}
  Soit $\eta>0$. Trouver $u_i \in \Hd$ tels que $\forall i \in \{1,2\}$
  \begin{equation}
    \forall v \in V, \quad \int_Y A \nabla u_i \nabla v + \eta \int_Y u_i v = - \int_Y A e_i \nabla v.
  \end{equation}
\end{pb}

À nouveau, on choisie un $i$ et on pose $u:=u_i$, 
\[
  a(u,v) = \int_Y A \nabla u \nabla v + \eta \int_Y u v, \qquad
  l(v) = -\int_Y A e_i \nabla v
\]

On note qu'en prenant $\phi=1$, on montre que $u$ est d'intrale nulle. Donc si $u\in\Hd$ résout le problème  \autoref{pb:celleta} alors $u\in V$, et respecte donc Poincaré.  \\
On remarque aussi que $l(v)$ reste la même que prédemment, il n'est donc pas nécessaire de refaire les calculs. 

Montrons que $a$ est toujours continue sous cette forme :
\begin{align*}
  \big|a(u,v)\big| &\leq \left| \int_Y A \nabla u \nabla v \right| + \eta \left| \int_Y  u v \right| && \mbox{ineg. triang} \\
                   &\leq \norm{A\nabla u}_{L^2} \norm{\nabla v}_{L^2} + \eta \norm{u}_{L^2} \norm{v}_{L^2} && \mbox{Cauchy Schwarz} \\
                   &\leq \beta \norm{\nabla u}_{L^2} \norm{\nabla v}_{L^2} + \eta \norm{u}_{L^2} \norm{v}_{L^2} && \mbox{A bornée} \\
                   &\leq \beta \norm{u}_{\Hd} \norm{v}_{\Hd} + \eta \norm{u}_{\Hd} \norm{v}_{\Hd} && \norm{\cdot}_{L^2}<\norm{\cdot}_{\Hd} \\
                   &\leq (\beta + \eta) \norm{u}_{\Hd} \norm{v}_{\Hd}
\end{align*}

Montrons maintenant que $a$ est toujours coercive
\begin{align*}
  a(u,u) &= \int_Y A \nabla u  \nabla u + \eta \int_Y u^2 \\
         &\geq \xi \norm{\nabla u}^2_{L^2} + \eta \norm{u}^2_{L^2} \\
         &\geq \zeta \norm{u}^2_{\Hd} + \eta \norm{ u}^2_{L^2} && \text{Poincaré dans }V\\
    \big(&\geq \zeta  \norm{u}^2_{\Hd}\big)
\end{align*} 
avec $\zeta = \frac{\xi}{(C^2+1)}$. Quand $\eta$ tend vers zéro on retrouve le resultat de la question 1. En considérant $V = \text{Vect}\big(\{\omega_I\}_{1,N})$, la matrice élements finis $\Am^{\eta}$ qu'on écrit donc
\[
  \Am^{\eta} = \K + \eta \M 
\]
avec
\[
  \K \in \R^{N\times N},\quad \K_{IJ} = \int A \nabla \omega_I \nabla \omega_J,  \\
  \M \in \R^{N\times N},\quad \M_{IJ} = \int \omega_I \omega_J
\]
tend vers la matrice $\K$ quand $\eta$ tend vers $0$ avec une vitesse \emph{a priori} proportionelle à $\eta$.

Il nous faut maintenant prouver que le problème dégradé est équivalent au problème de cellule initale quand $\eta$ tend vers zéro. On va précisément démontrer
que $\quad\forall i \in \{1,2\}, \exists C>0 ~tq~ \norm{\omega_i-\omega_i^\eta}_{H^1} < C \eta$.
En sommant les équations des problèmes \autoref{pb:cell} et \autoref{pb:celleta}, on a
\[
  \int_Y A \nabla \big(\omega_i-\omega_i^\eta\big) \nabla \phi = \eta \int_Y \omega_i^\eta \phi
\]
or $\omega_i-\omega_i^\eta\in V$, donc on peut choisir $\phi=\omega_i-\omega_i^\eta$ :
\begin{align*}
  \int_Y A \big(\nabla \big(\omega_i-\omega_i^\eta\big)\big)^2
  &= \eta \int_Y \omega_i^\eta \big(\omega_i-\omega_i^\eta\big) \\
  \left| \int_Y A \nabla \big(\omega_i-\omega_i^\eta\big) \nabla \big(\omega_i-\omega_i^\eta\big) \right|
  &= \left| \eta \int_Y \omega_i^\eta \big(\omega_i-\omega_i^\eta\big) \right| && \text{valeur absolue} \\
  \xi \norm{\nabla \big(\omega_i-\omega_i^\eta \big)}_{L^2}^2
  &\leq \eta \norm{\omega_i^\eta}_{L^2} \norm{\omega_i-\omega_i^\eta}_{L^2} && \text{Cauchy-Schwarz et $A$ $\xi$-coercitive} \\
  \xi (D^2+1) \norm{\omega_i-\omega_i^\eta }_{H^1}^2 &\leq \eta \norm{\omega_i^\eta}_{H^1} \norm{\omega_i-\omega_i^\eta }_{H^1} && \text{Poincaré et }\norm{\cdot}_{L^2}<\norm{\cdot}_{\Hd} \\
  \norm{\omega_i-\omega_i^\eta }_{H^1} &\leq \eta \frac{\norm{\omega_i^\eta}_{H^1}}{\xi (D^2+1)}
\end{align*}

On peut montrer que $\norm{\omega_i^\eta}_{H^1}$ est majorée en prenant l'équation du problème \autoref{pb:celleta} avec $\phi=\omega_i^\eta$ :
\begin{align*}
  \left| \int_Y A \nabla \omega_i^\eta \nabla \omega_i^\eta  + \eta \int_Y \big(\omega_i^\eta\big)^2 \right|  = \left|\int_Y A e_i \nabla \omega_i^\eta\right| \\
  \left| \int_Y A \nabla \omega_i^\eta \nabla \omega_i^\eta \right| \leq \left|\int_Y A e_i \nabla \omega_i^\eta\right| && \text{Car }\eta\int_Y \big(\omega_i^\eta\big)^2 \geq  0 \\
  \xi \norm{\nabla \omega_i^\eta}_{L^2}^2 \leq \norm{A e_i}_{L^2} \norm{ \omega_i^\eta}_{L^2} && \text{Cauchy-Schwarz et $A$ $\xi$-coercitive} \\
  (D^2+1) \xi \norm{\omega_i^\eta}_{H^1}^2 \leq \norm{A e_i}_{L^2}\norm{\omega_i^\eta}_{H^1} && \text{Poincaré et }\norm{\cdot}_{L^2}<\norm{\cdot}_{\Hd}\\
  \norm{\omega_i^\eta}_{H^1} \leq \frac{1}{(D^2+1) \xi}\norm{A e_i}_{L^2}
\end{align*}


\subsection{Discrétisation des problèmes de cellule}

\question{4}{Assembler la matrice EF}

\question{5}{Calculer le second membre en se rappelant que $\nabla y_i=e_i$.}
On peut réécrire le second membre avec cette nouvelle propriété
\[
  \int_Y \big(A(y)e_i, \nabla \phi\big) = \int_Y \big(A(y)\nabla y_i, \nabla \phi\big) 
                                        = a(y_i, \phi)
\]
avec le $a$ défini comme pour l'\autoref{eq:pbcell}. On pourra donc reprendre la matrice $\K$ pour calculer les vecteurs second membre $L_i$ des équations discrétisées pour les différents problèmes de cellule :
\[
  L_i = (y_i^T\K)^T \quad \forall i \in \{1,2\}.
\]

\subsection{Première validation}

\question{7}{Trouver une solution exacte à \autoref{eq:pbcell} avec $A=Id$.}
On a directement que
\begin{align*}
  &\int_Y \big(\nabla \omega_i+e_i\big) \nabla \phi = 0  \\
  \iff&\int_{\partial Y} \big(\nabla \omega_i+e_i\big) \phi - \int_Y \nabla \big(\nabla \omega_i+e_i\big) \phi = 0 && \text{IPP} \\
  \iff&\int_Y \Delta \omega_i \phi = 0 && \int_{\partial Y}...=0\text{ car tout est périodique\footnote{Plus facile à voir en 1D mais marche aussi en 2D.}} \\
  \iff&\Delta\omega_i = 0 && \forall i \in \{1,2\}
\end{align*}
Pour résoudre cette équation on remarque la fonction nulle est solution. Or le problème \autoref{eq:pbcell} étant bien posé, la solution est unique, donc on a bien
\[
  \omega_i = 0, \quad \forall i \in\{1,2\}.
\]

\question{8}{De même pour
$A = \left(\begin{matrix}
  1 & 0 \\
  0 & 2 
\end{matrix}\right)$.
}
En applicant le même processus que précedemment,
\begin{align*}
  &\int_Y \big(A(\nabla \omega_i+e_i)\big) \nabla \phi = 0  \\
  \iff&\int_Y \nabla \big(A(\nabla \omega_i+e_i)\big) \phi = 0 \\
  \iff&\int_Y \nabla \big(A(\nabla \omega_i)\big) \phi = 0 \qquad \forall \phi \in V\\
  \iff&\nabla \big(A(\nabla \omega_i)\big)=0 \qquad \forall i \in \{1,2\}.
\end{align*}
Avec la bonne forme de $A$ on a 
\[
  \partial^2_{y_1} \omega_i + 2\partial^2_{y_2} \omega_i = 0 
\]
dont la fonction nulle est encore solution, et donc on a 
$
  \omega_i = 0~~\forall i \in\{1,2\}.
$

\noindent
\textbf{Remarque} : il en va de même de pour tout $A$ ne dépendant pas de $y$, car il faut que
$
  \nabla \omega_i = 0~~\forall i \in\{1,2\}
$
pour que l'\autoref{eq:Ah} donne $\Ah = A$.

\begin{equation}
  \label{eq:Ah}
  \Ah_{jk} = \int_Y \big(A(y)(e_k+\nabla \omega_k(y)), e_j+\nabla \omega_j(y)\big) dy \qquad \forall 1<i,j<2
\end{equation}

\question{9}{A l'aide de matrices déjà assemblées, calculer la matrice $\Ah$.}
On peut en effet réutiliser la matrice de rigiditée précedemment constuite en remarquant que :
\begin{align*}
  \Ah_{jk} &= a(y_k+\omega_k, y_j+\omega_j) \\
  \Ah_{jk} &= (Y_k + W_k)^T \K (Y_j + W_j) \qquad \forall 1<i,j<2
\end{align*}
avec pour tout sommet $S(j), \quad 1<j<N $
\[
  (Y_i)_j = S(j)_{y_i} 
\]
\[
  (W_i)_j = \omega_i(S(j))
\]
\end{document}

%%% Local Variables:
%%% mode: latex
%%% TeX-master: t
%%% End:

\message{ !name(rapport.tex) !offset(-400) }
