\message{ !name(rapport.tex)}\documentclass[11pt]{article}
\usepackage[utf8]{inputenc}

% MATH
\usepackage{amssymb}
\usepackage{amsmath}
\usepackage{amsthm}
\usepackage{mathtools}

%\newtheorem{cusdef}{Def}
%\newtheorem{custhm}{Thm}
%\newtheorem{cuscor}{Cor}
%\newtheorem{cusrk}{Rk}
\newenvironment{cusenv}[2]
{\begin{samepage}\noindent\textbf{#1} -- (#2) \par}
{\end{samepage} \bigskip}

\newenvironment{cusdef}[1]
{\begin{cusenv}{Def}{#1}}{\end{cusenv}}

\newenvironment{custhm}[1]
{\begin{cusenv}{Thm}{#1}}{\end{cusenv}}

\newenvironment{cuscor}[1]
{\begin{cusenv}{Cor}{#1}}{\end{cusenv}}

\newenvironment{cusprop}[1]
{\begin{cusenv}{Prop}{#1}}{\end{cusenv}}

\newenvironment{cusrk}[1]
{\begin{cusenv}{Rk}{#1}}{\end{cusenv}}	

% MISE EN FORME
\usepackage[left=2cm,right=2cm,top=2cm,bottom=2cm]{geometry}

\usepackage{hyperref}
\usepackage{xcolor}
\hypersetup{
	colorlinks,
	linkcolor={red!50!black},
	citecolor={blue!50!black},
	urlcolor={blue!80!black}
}


% MACROS
\usepackage{xparse}
\newcommand{\smbox}[1]{\mbox{\footnotesize #1}}
\newcommand{\A}{\mathcal{A}}
\newcommand{\Am}{\mathbb{A}}
\newcommand{\N}{\mathbb{N}}
\newcommand{\C}{\mathbb{C}}
\renewcommand{\P}{\mathbb{P}}
\newcommand{\R}{\mathbb{R}}
\newcommand{\K}{\mathbb{K}}
\newcommand{\M}{\mathbb{M}}
\newcommand{\topo}{\mathcal{T}}
\newcommand{\Lin}{\mathcal{L}}
\newcommand{\Int}{\mbox{Int}\,}
\newcommand{\st}{~\mbox{s.t.}~}
\newcommand{\miff}{~\mbox{iff}~}
\newcommand{\ie}{\emph{i.e.} }
\newcommand{\ms}{~~~}
\newcommand{\norm}[1]{\left|\left|#1\right|\right|}
\newcommand{\dual}[1]{#1'}
\newcommand{\sca}[2]{\big<#1, #2\big>}
\newcommand{\cont}[1]{\mathcal{C}^{#1}}
\newcommand{\question}[2]{\paragraph{Question #1.}\textit{#2} \\}
\newcommand{\Hd}{H^1_{\#}}



\title{Rapport TP X01 \\ TP3}
\author{Aurélien Valade}
\date{}

\begin{document}

\message{ !name(rapport.tex) !offset(20) }
\section{Solution au problème homogénéisé}

\subsection{Les problèmes de cellule}

\question{1}{Montrer que le problème suivant est bien posé:}

Trouver $u \in V$ tel que
\begin{equation}
  \label{eq:pbcell}
  \forall v \in V, \quad \int_Y A \nabla u \nabla v = - \int_Y A e_i \nabla v
\end{equation}
avec
\begin{equation}
  V = 
  \left\{
    \psi \in \Hd(Y), \quad \int_Y \psi = 0
  \right\}
\end{equation}
On pose
\begin{equation}
  a(u,v) = \int_Y A \nabla u \nabla v, \quad
  l(v) = -\int_Y A e_i \nabla v
\end{equation}

On a bien $a(u,v)$ et $l(v)$ (bi)linéaires. Montrons que $a(u,v)$ continue 
\begin{align}
  \label{eq:ac}
  \big|a(u,v)\big| &= \left| \int_Y A \nabla u \nabla v \right| \\
                   &\leq \norm{A\nabla u}_{L^2} \norm{\nabla v}_{L^2} && \mbox{Cauchy Schwarz} \\
                   &\leq \beta \norm{\nabla u}_{L^2} \norm{\nabla v}_{L^2} && \mbox{A bornée} \\
                   &\leq \beta \norm{u}_{\Hd} \norm{v}_{\Hd} && \norm{\cdot}_{L^2}<\norm{\cdot}_{\Hd} \\
\end{align}

Montrons que $l(v)$ est continue
\begin{align}
  \label{eq:ac}
  \big|l(v)\big| &= \left| \int_Y A e_i \nabla v \right|  && \text{définition} \\
                 &\leq \norm{Ae_i}_{L^2} \norm{\nabla v}_{L^2} && \text{Cauchy Schwarz} \\
                 &\leq \beta_i \norm{\nabla v}_{L^2} && \text{$A$ bornée, donc $A e_i$ aussi} \\
                 &\leq \eta_i \norm{v}_{\Hd} && \norm{\cdot}_{L^2}<\norm{\cdot}_{\Hd}
\end{align}

Montrons que $a(u,v)$ est coercive
\begin{align}
  \label{eq:co}
  a(u,u) &= \int_Y A \nabla u  \nabla u \\
         &\geq \xi \int_Y \nabla u^2  && \text{$A$ minorée par $\xi$}\\
         &\geq \xi \norm{\nabla u}^2_{L^2} \\
         &\geq \zeta \norm{u}^2_{\Hd} && \text{Pointcaré dans $V$} 
\end{align}
avec $\zeta = \frac{\xi}{(C^2+1)}$, $C$ étant la constante de poincaré associée à $V$.


\question{2}{De même pour le problème modifié}

Soit $\eta>0$. Trouver $u \in V$ tel que
\begin{equation}
  \label{eq:pbcelleta}
  \forall v \in V, \quad \int_Y A \nabla u \nabla v + \eta \int_Y u v = - \int_Y A e_i \nabla v
\end{equation}
On pose
\begin{equation}
  a(u,v) = \int_Y A \nabla u \nabla v + \eta \int_Y u v \quad
  l(v) = -\int_Y A e_i \nabla v
\end{equation}

On remarque que $l(v)$ reste la même que dans la question 1, il n'est donc pas nécéssaire de refaire les calculs.

Montrons que $a$ est toujours continue sous cette forme :
\begin{align}
  \label{eq:ac}
  \big|a(u,v)\big| &\leq \left| \int_Y A \nabla u \nabla v \right| + \eta \left| \int_Y  u v \right| && \mbox{ineg. triang} \\
                   &\leq \norm{A\nabla u}_{L^2} \norm{\nabla v}_{L^2} + \eta \norm{u}_{L^2} \norm{v}_{L^2} && \mbox{Cauchy Schwarz} \\
                   &\leq \beta \norm{\nabla u}_{L^2} \norm{\nabla v}_{L^2} + \eta \norm{u}_{L^2} \norm{v}_{L^2} && \mbox{A bornée} \\
                   &\leq \beta \norm{u}_{\Hd} \norm{v}_{\Hd} + \eta \norm{u}_{\Hd} \norm{v}_{\Hd} && \norm{\cdot}_{L^2}<\norm{\cdot}_{\Hd} \\
                   &\leq (\beta + \eta) \norm{u}_{\Hd} \norm{v}_{\Hd}
\end{align}

Montrons maintenant que $a$ est toujours coercive
\begin{align}
  \label{eq:co}
  a(u,u) &= \int_Y A \nabla u  \nabla u + \eta \int_Y u^2 \\
         &\geq \xi \norm{\nabla u}^2_{L^2} + \eta \norm{u}^2_{L^2} \\
         &\geq \zeta \norm{u}^2_{\Hd} + \eta \norm{ u}^2_{L^2} && \text{Poincaré dans }V\\
    \big(&\geq \zeta  \norm{u}^2_{\Hd}\big)
\end{align} 
avec $\zeta = \frac{\xi}{(C^2+1)}$. Quand $\eta$ tend vers zéro on retrouve le resultat de la question 1. En considérant $V = \text{Vect}\big(\{\omega_I\}_{1,N})$, la matrice élements finis $\Am^{\eta}$ qu'on écrit donc
\begin{equation}
  \Am^{\eta} = \K + \eta \M 
\end{equation}
avec
\begin{align}
  &\K \in \R^{N\times N},\quad \K_{IJ} = \int A \nabla \omega_I \nabla \omega_J,  \\
  &\M \in \R^{N\times N},\quad \M_{IJ} = \int \omega_I \omega_J
\end{align}
tend vers la matrice $\K$ quand $\eta$ tend vers $0$ avec une vitesse \emph{a priori} proportionelle à $\eta$.

\question{3}{Montrer que $\quad\forall i \in \{1,2\}, \exists C>0 ~tq~ \norm{\omega_i-\omega_i^\eta}_{H^1} < C \eta$.}
En sommant \autoref{eq:pbcell} et \autoref{eq:pbcelleta}, on a
\begin{equation}
  \int_Y A \nabla \big(\omega_i-\omega_i^\eta\big) \nabla \phi = \eta \int_Y \omega_i^\eta \phi
\end{equation}
or $\omega_i-\omega_i^\eta\in\Hd$, donc on peut choisir $\phi=\omega_i-\omega_i^\eta$ :
\begin{align}
  \int_Y A \big(\nabla \big(\omega_i-\omega_i^\eta\big)\big)^2
  &= \eta \int_Y \omega_i^\eta \big(\omega_i-\omega_i^\eta\big) \\
  \left| \int_Y A \nabla \big(\omega_i-\omega_i^\eta\big) \nabla \big(\omega_i-\omega_i^\eta\big) \right|
  &= \left| \eta \int_Y \omega_i^\eta \big(\omega_i-\omega_i^\eta\big) \right| && \text{valeur absolue} \\
  \xi \norm{\nabla \big(\omega_i-\omega_i^\eta \big)}_{L^2}^2
  &\leq \eta \norm{\omega_i^\eta}_{L^2} \norm{\big(\omega_i-\omega_i^\eta \big)}_{L^2} && \text{Cauchy-Schwarz et $A$ $\xi$-coercitive} \\
  \xi (D^2+1) \norm{\big(\omega_i-\omega_i^\eta \big)}_{H^1}^2 &\leq \eta \norm{\omega_i^\eta}_{H^1} \norm{\big(\omega_i-\omega_i^\eta \big)}_{H^1} && \text{Poincaré et }\norm{\cdot}_{L^2}<\norm{\cdot}_{\Hd} \\
  \norm{\big(\omega_i-\omega_i^\eta \big)}_{H^1} &\leq \eta \frac{\norm{\omega_i^\eta}_{H^1}}{\xi (D^2+1)}
\end{align}

Si on borne $\eta$ par le haut, c'est à dire qu'on travaille sur un interval fini $I_\eta \subset  \R$, comme le problème est bien posé pour tous ces $\eta$, on pose
\begin{equation}
  \gamma = \sup_{\eta\in I_\eta}\norm{\omega_i^\eta}_{H^1}
\end{equation}
et on récupère 
\begin{equation}
  \norm{\big(\omega_i-\omega_i^\eta \big)}_{H^1} \leq C \eta 
\end{equation}
avec $C = \frac{\gamma}{\xi (D^2+1)}$.

\subsection{Discrétisation des problèmes de cellule}

\question{4}{Assembler la matrice EF}

\question{5}{Calculer le second membre en se rappelant que $\nabla y_i=e_i$.}

On peut réécrire le second membre avec cette nouvelle propriété
\begin{align}
  \int_Y \big(A(y)e_i, \nabla \phi\big) &= \int_Y \big(A(y)\nabla y_i, \nabla \phi\big) \\
                                        &= a(y_i, \phi)
\end{align}
avec le $a$ défini comme pour l'\autoref{eq:pbcell}. On pourra donc reprendre la méthode la construction de $\K$ pour calculer les vecteurs second membre $L_i$ des équations discrétisées pour les différents problèmes de cellule
\begin{equation}
  (L_i)_j = a(y_i, \omega_j) \quad \forall j \in [1, N], \forall i \in \{1,2\}
\end{equation}

\subsection{Première validation}

\question{7}{Trouver une solution exacte à \autoref{eq:pbcell} avec $A=Id$.}
On a directement que
\begin{align}
  &\int_Y \big(\nabla \omega_i+e_i\big) \nabla \phi = 0  \\
  \iff&\int_{\partial Y} \big(\nabla \omega_i+e_i\big) \phi - \int_Y \nabla \big(\nabla \omega_i+e_i\big) \phi = 0 && \text{IPP} \\
  \iff&\int_Y \Delta \omega_i \phi = 0 && \int_{\partial Y}...=0\text{ car tout est périodique\footnote{Plus facile à voir en 1D mais marche aussi en 2D.}} \\
  \iff & \Delta\omega_i = 0 &&\quad \forall i \in \{1,2\}
\end{align}
Pour résoudre cette équation on peut passer dans l'espace de Fourrier


\question{8}{De même pour
$A = \left(\begin{matrix}
  1 & 0 \\
  0 & 2 
\end{matrix}\right)$.
}

Vu la forme de $A$, on a immédiatement que  $\omega_1$ reste inchangé. En revanche, on retrouve aisément $\omega_2 = -2 y_2 +1$.

\message{ !name(rapport.tex) !offset(17) }

\end{document}


%%% Local Variables:
%%% mode: latex
%%% TeX-master: t
%%% End:
