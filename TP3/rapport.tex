\documentclass[11pt]{article}
\usepackage[utf8]{inputenc}

% MATH
\usepackage{amssymb}
\usepackage{amsmath}
\usepackage{amsthm}
\usepackage{mathtools}

%\newtheorem{cusdef}{Def}
%\newtheorem{custhm}{Thm}
%\newtheorem{cuscor}{Cor}
%\newtheorem{cusrk}{Rk}
\newenvironment{cusenv}[2]
{\begin{samepage}\noindent\textbf{#1} -- (#2) \par}
{\end{samepage} \bigskip}

\newenvironment{cusdef}[1]
{\begin{cusenv}{Def}{#1}}{\end{cusenv}}

\newenvironment{custhm}[1]
{\begin{cusenv}{Thm}{#1}}{\end{cusenv}}

\newenvironment{cuscor}[1]
{\begin{cusenv}{Cor}{#1}}{\end{cusenv}}

\newenvironment{cusprop}[1]
{\begin{cusenv}{Prop}{#1}}{\end{cusenv}}

\newenvironment{cusrk}[1]
{\begin{cusenv}{Rk}{#1}}{\end{cusenv}}	

% MISE EN FORME
\usepackage[left=2cm,right=2cm,top=2cm,bottom=2cm]{geometry}

\usepackage{hyperref}
\usepackage{xcolor}
\hypersetup{
	colorlinks,
	linkcolor={red!50!black},
	citecolor={blue!50!black},
	urlcolor={blue!80!black}
}


% MACROS
\usepackage{xparse}
\newcommand{\smbox}[1]{\mbox{\footnotesize #1}}
\newcommand{\A}{\mathcal{A}}
\newcommand{\Am}{\mathbb{A}}
\newcommand{\N}{\mathbb{N}}
\newcommand{\C}{\mathbb{C}}
\renewcommand{\P}{\mathbb{P}}
\newcommand{\R}{\mathbb{R}}
\newcommand{\K}{\mathbb{K}}
\newcommand{\M}{\mathbb{M}}
\newcommand{\Ah}{A^{hom}}

\newcommand{\ie}{\emph{i.e.} }
\newcommand{\ms}{~~~}
\newcommand{\norm}[1]{\left|\left|#1\right|\right|}
\newcommand{\dual}[1]{#1'}
\newcommand{\sca}[2]{\big<#1, #2\big>}
\newcommand{\cont}[1]{\mathcal{C}^{#1}}
\newcommand{\question}[2]{\paragraph{Question #1.}\textit{#2} \\}
\newcommand{\Hd}{H^1_{\#}}



\title{Rapport TP X01 \\ TP3}
\author{Aurélien Valade}
\date{}

\begin{document}
\maketitle

\section{Intruction et structure du code}


Le but de ce TP est de résoudre le problème suivant : trouver $u \in H^1_0(\Omega)$ tel que  
\begin{equation}
  \begin{cases}
    \nabla \big(A(x,y) \nabla u\big) = f \quad \mbox{sur}\quad \Omega\\
    u = 0 \quad \mbox{sur}\quad \partial\Omega.
  \end{cases}
\end{equation}

La structure du code ainsi que l'arrangement des dossiers ont été un peu modifiés. Tous les \texttt{*.msh} se trouvent dans le dossier \texttt{geoms/} avec un executable \texttt{bash} pour en créer à volonté.

De plus, le corps de la routine principale se trouve maintenant dans \texttt{principal\_dirichlet\_aux.m}, cependant le fichier script est toujours bien \texttt{principal\_dirichlet.m}. 

Un code de calcul formel rédigé en python a été ajouté. 

\section{Solution exacte}


\section{Solution au problème homogénéisé}

\subsection{Les problèmes de cellule}

\question{1}{Montrer que le problème suivant est bien posé:}
Trouver $u \in V$ tel que
\begin{equation}
  \label{eq:pbcell}
  \forall v \in V, \quad \int_Y A \nabla u \nabla v = - \int_Y A e_i \nabla v
\end{equation}
avec
\[
  V = 
  \left\{
    \psi \in \Hd(Y), \qquad \int_Y \psi = 0
  \right\}
\]
On pose
\[
  (u,v) = \int_Y A \nabla u \nabla v, \quad
  l(v) = -\int_Y A e_i \nabla v
\]

On a bien $a(u,v)$ et $l(v)$ (bi)linéaires. Montrons que $a(u,v)$ continue 
\begin{align*}
  \big|a(u,v)\big| &= \left| \int_Y A \nabla u \nabla v \right| \\
                   &\leq \norm{A\nabla u}_{L^2} \norm{\nabla v}_{L^2} && \mbox{Cauchy Schwarz} \\
                   &\leq \beta \norm{\nabla u}_{L^2} \norm{\nabla v}_{L^2} && \mbox{A bornée} \\
                   &\leq \beta \norm{u}_{\Hd} \norm{v}_{\Hd} && \norm{\cdot}_{L^2}<\norm{\cdot}_{\Hd} \\
\end{align*}

Montrons que $l(v)$ est continue
\begin{align*}
  \big|l(v)\big| &= \left| \int_Y A e_i \nabla v \right|  && \text{définition} \\
                 &\leq \norm{Ae_i}_{L^2} \norm{\nabla v}_{L^2} && \text{Cauchy Schwarz} \\
                 &\leq \beta_i \norm{\nabla v}_{L^2} && \text{$A$ bornée, donc $A e_i$ aussi} \\
                 &\leq \eta_i \norm{v}_{\Hd} && \norm{\cdot}_{L^2}<\norm{\cdot}_{\Hd}
\end{align*}

Montrons que $a(u,v)$ est coercive
\begin{align*}
  a(u,u) &= \int_Y A \nabla u  \nabla u \\
         &\geq \xi \int_Y \nabla u^2  && \text{$A$ minorée par $\xi$}\\
         &\geq \xi \norm{\nabla u}^2_{L^2} \\
         &\geq \zeta \norm{u}^2_{\Hd} && \text{Pointcaré dans $V$} 
\end{align*}
avec $\zeta = \frac{\xi}{(C^2+1)}$, $C$ étant la constante de poincaré associée à $V$.


\question{2}{De même pour le problème modifié}
Soit $\eta>0$. Trouver $u \in \Hd$ tel que
\begin{equation}
  \label{eq:pbcelleta}
  \forall v \in V, \quad \int_Y A \nabla u \nabla v + \eta \int_Y u v = - \int_Y A e_i \nabla v
\end{equation}
On pose
\[
  a(u,v) = \int_Y A \nabla u \nabla v + \eta \int_Y u v, \qquad
  l(v) = -\int_Y A e_i \nabla v
\]

On note qu'en prenant $\phi=1$, on montre que $u$ est d'intrale nulle. Donc si $u\in\Hd$ résout \autoref{eq:pbcelleta} alors $u\in V$, et respecte donc Poincaré.  \\
On remarque aussi que $l(v)$ reste la même que dans la question 1, il n'est donc pas nécéssaire de refaire les calculs. 

Montrons que $a$ est toujours continue sous cette forme :
\begin{align*}
  \big|a(u,v)\big| &\leq \left| \int_Y A \nabla u \nabla v \right| + \eta \left| \int_Y  u v \right| && \mbox{ineg. triang} \\
                   &\leq \norm{A\nabla u}_{L^2} \norm{\nabla v}_{L^2} + \eta \norm{u}_{L^2} \norm{v}_{L^2} && \mbox{Cauchy Schwarz} \\
                   &\leq \beta \norm{\nabla u}_{L^2} \norm{\nabla v}_{L^2} + \eta \norm{u}_{L^2} \norm{v}_{L^2} && \mbox{A bornée} \\
                   &\leq \beta \norm{u}_{\Hd} \norm{v}_{\Hd} + \eta \norm{u}_{\Hd} \norm{v}_{\Hd} && \norm{\cdot}_{L^2}<\norm{\cdot}_{\Hd} \\
                   &\leq (\beta + \eta) \norm{u}_{\Hd} \norm{v}_{\Hd}
\end{align*}

Montrons maintenant que $a$ est toujours coercive
\begin{align*}
  a(u,u) &= \int_Y A \nabla u  \nabla u + \eta \int_Y u^2 \\
         &\geq \xi \norm{\nabla u}^2_{L^2} + \eta \norm{u}^2_{L^2} \\
         &\geq \zeta \norm{u}^2_{\Hd} + \eta \norm{ u}^2_{L^2} && \text{Poincaré dans }V\\
    \big(&\geq \zeta  \norm{u}^2_{\Hd}\big)
\end{align*} 
avec $\zeta = \frac{\xi}{(C^2+1)}$. Quand $\eta$ tend vers zéro on retrouve le resultat de la question 1. En considérant $V = \text{Vect}\big(\{\omega_I\}_{1,N})$, la matrice élements finis $\Am^{\eta}$ qu'on écrit donc
\[
  \Am^{\eta} = \K + \eta \M 
\]
avec
\[
  \K \in \R^{N\times N},\quad \K_{IJ} = \int A \nabla \omega_I \nabla \omega_J,  \\
  \M \in \R^{N\times N},\quad \M_{IJ} = \int \omega_I \omega_J
\]
tend vers la matrice $\K$ quand $\eta$ tend vers $0$ avec une vitesse \emph{a priori} proportionelle à $\eta$.

\question{3}{Montrer que $\quad\forall i \in \{1,2\}, \exists C>0 ~tq~ \norm{\omega_i-\omega_i^\eta}_{H^1} < C \eta$.}
En sommant \autoref{eq:pbcell} et \autoref{eq:pbcelleta}, on a
\[
  \int_Y A \nabla \big(\omega_i-\omega_i^\eta\big) \nabla \phi = \eta \int_Y \omega_i^\eta \phi
\]
or $\omega_i-\omega_i^\eta\in V$, donc on peut choisir $\phi=\omega_i-\omega_i^\eta$ :
\begin{align*}
  \int_Y A \big(\nabla \big(\omega_i-\omega_i^\eta\big)\big)^2
  &= \eta \int_Y \omega_i^\eta \big(\omega_i-\omega_i^\eta\big) \\
  \left| \int_Y A \nabla \big(\omega_i-\omega_i^\eta\big) \nabla \big(\omega_i-\omega_i^\eta\big) \right|
  &= \left| \eta \int_Y \omega_i^\eta \big(\omega_i-\omega_i^\eta\big) \right| && \text{valeur absolue} \\
  \xi \norm{\nabla \big(\omega_i-\omega_i^\eta \big)}_{L^2}^2
  &\leq \eta \norm{\omega_i^\eta}_{L^2} \norm{\omega_i-\omega_i^\eta}_{L^2} && \text{Cauchy-Schwarz et $A$ $\xi$-coercitive} \\
  \xi (D^2+1) \norm{\omega_i-\omega_i^\eta }_{H^1}^2 &\leq \eta \norm{\omega_i^\eta}_{H^1} \norm{\omega_i-\omega_i^\eta }_{H^1} && \text{Poincaré et }\norm{\cdot}_{L^2}<\norm{\cdot}_{\Hd} \\
  \norm{\omega_i-\omega_i^\eta }_{H^1} &\leq \eta \frac{\norm{\omega_i^\eta}_{H^1}}{\xi (D^2+1)}
\end{align*}

On peut montrer que $\norm{\omega_i^\eta}_{H^1}$ est majorée en prenant l'\autoref{eq:pbcelleta} avec $\phi=\omega_i^\eta$ :
\begin{align*}
  \left| \int_Y A \nabla \omega_i^\eta \nabla \omega_i^\eta  + \eta \int_Y \big(\omega_i^\eta\big)^2 \right|  = \left|\int_Y A e_i \nabla \omega_i^\eta\right| \\
  \left| \int_Y A \nabla \omega_i^\eta \nabla \omega_i^\eta \right| \leq \left|\int_Y A e_i \nabla \omega_i^\eta\right| && \text{Car }\eta\int_Y \big(\omega_i^\eta\big)^2 \geq  0 \\
  \xi \norm{\nabla \omega_i^\eta}_{L^2}^2 \leq \norm{A e_i}_{L^2} \norm{ \omega_i^\eta}_{L^2} && \text{Cauchy-Schwarz et $A$ $\xi$-coercitive} \\
  (D^2+1) \xi \norm{\omega_i^\eta}_{H^1}^2 \leq \norm{A e_i}_{L^2}\norm{\omega_i^\eta}_{H^1} && \text{Poincaré et }\norm{\cdot}_{L^2}<\norm{\cdot}_{\Hd}\\
  \norm{\omega_i^\eta}_{H^1} \leq \frac{1}{(D^2+1) \xi}\norm{A e_i}_{L^2}
\end{align*}
\subsection{Discrétisation des problèmes de cellule}

\question{4}{Assembler la matrice EF}

\question{5}{Calculer le second membre en se rappelant que $\nabla y_i=e_i$.}
On peut réécrire le second membre avec cette nouvelle propriété
\[
  \int_Y \big(A(y)e_i, \nabla \phi\big) = \int_Y \big(A(y)\nabla y_i, \nabla \phi\big) 
                                        = a(y_i, \phi)
\]
avec le $a$ défini comme pour l'\autoref{eq:pbcell}. On pourra donc reprendre la matrice $\K$ pour calculer les vecteurs second membre $L_i$ des équations discrétisées pour les différents problèmes de cellule :
\[
  L_i = (y_i^T\K)^T \quad \forall i \in \{1,2\}.
\]

\subsection{Première validation}

\question{7}{Trouver une solution exacte à \autoref{eq:pbcell} avec $A=Id$.}
On a directement que
\begin{align*}
  &\int_Y \big(\nabla \omega_i+e_i\big) \nabla \phi = 0  \\
  \iff&\int_{\partial Y} \big(\nabla \omega_i+e_i\big) \phi - \int_Y \nabla \big(\nabla \omega_i+e_i\big) \phi = 0 && \text{IPP} \\
  \iff&\int_Y \Delta \omega_i \phi = 0 && \int_{\partial Y}...=0\text{ car tout est périodique\footnote{Plus facile à voir en 1D mais marche aussi en 2D.}} \\
  \iff&\Delta\omega_i = 0 && \forall i \in \{1,2\}
\end{align*}
Pour résoudre cette équation on remarque la fonction nulle est solution. Or le problème \autoref{eq:pbcell} étant bien posé, la solution est unique, donc on a bien
\[
  \omega_i = 0, \quad \forall i \in\{1,2\}.
\]

\question{8}{De même pour
$A = \left(\begin{matrix}
  1 & 0 \\
  0 & 2 
\end{matrix}\right)$.
}
En applicant le même processus que précedemment,
\begin{align*}
  &\int_Y \big(A(\nabla \omega_i+e_i)\big) \nabla \phi = 0  \\
  \iff&\int_Y \nabla \big(A(\nabla \omega_i+e_i)\big) \phi = 0 \\
  \iff&\int_Y \nabla \big(A(\nabla \omega_i)\big) \phi = 0 \qquad \forall \phi \in V\\
  \iff&\nabla \big(A(\nabla \omega_i)\big)=0 \qquad \forall i \in \{1,2\}.
\end{align*}
Avec la bonne forme de $A$ on a 
\[
  \partial^2_{y_1} \omega_i + 2\partial^2_{y_2} \omega_i = 0 
\]
dont la fonction nulle est encore solution, et donc on a 
$
  \omega_i = 0~~\forall i \in\{1,2\}.
$

\noindent
\textbf{Remarque} : il en va de même de pour tout $A$ ne dépendant pas de $y$, car il faut que
$
  \nabla \omega_i = 0~~\forall i \in\{1,2\}
$
pour que l'\autoref{eq:Ah} donne $\Ah = A$.

\begin{equation}
  \label{eq:Ah}
  \Ah_{jk} = \int_Y \big(A(y)(e_k+\nabla \omega_k(y)), e_j+\nabla \omega_j(y)\big) dy \qquad \forall 1<i,j<2
\end{equation}

\question{9}{A l'aide de matrices déjà assemblées, calculer la matrice $\Ah$.}
On peut en effet réutiliser la matrice de rigiditée précedemment constuite en remarquant que :
\begin{align*}
  \Ah_{jk} &= a(y_k+\omega_k, y_j+\omega_j) \\
  \Ah_{jk} &= (Y_k + W_k)^T \K (Y_j + W_j) \qquad \forall 1<i,j<2
\end{align*}
avec pour tout sommet $S(j), \quad 1<j<N $
\[
  (Y_i)_j = S(j)_{y_i} 
\]
\[
  (W_i)_j = \omega_i(S(j))
\]
\end{document}

%%% Local Variables:
%%% mode: latex
%%% TeX-master: t
%%% End:
