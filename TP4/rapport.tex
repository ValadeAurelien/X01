\documentclass[11pt]{article}
\usepackage[utf8]{inputenc}

% MATH
\usepackage{amssymb}
\usepackage{amsmath}
\usepackage{amsthm}
\usepackage{mathtools}
\usepackage{bm}

\newtheorem{pb}{Problème}
\newtheorem{rmq}{Remarque}

% MISE EN FORME
\usepackage[left=2cm,right=2cm,top=2cm,bottom=2cm]{geometry}

\usepackage{hyperref}
\usepackage{xcolor}
\hypersetup{
	colorlinks,
	linkcolor={red!50!black},
	citecolor={blue!50!black},
	urlcolor={blue!80!black}
}


% MACROS
\usepackage{xparse}
\newcommand{\smbox}[1]{\mbox{\footnotesize #1}}
\newcommand{\Am}{\mathbb{A}}
\newcommand{\N}{\mathbb{N}}
\newcommand{\C}{\mathbb{C}}
\renewcommand{\P}{\mathbb{P}}
\newcommand{\R}{\mathbb{R}}
\newcommand{\M}{\mathbb{M}}
\newcommand{\D}{\mathbb{D}}
\newcommand{\tD}{\widetilde{\mathbb{D}}}
\newcommand{\J}{\mathbb{J}}
\newcommand{\K}{\mathbb{K}}
\newcommand{\Kk}{\mathbb{K}^k}
\newcommand{\Ktk}{\mathbb{K}^{T,k}}
\newcommand{\Ktkc}{\mathbb{K}^{T, k}_c}
\newcommand{\T}{\mathcal{T}}
\newcommand{\Ah}{A^{hom}}
\newcommand{\uh}{u^{hom}}
\newcommand{\bx}{\bar{x}}
\newcommand{\tphi}{\tilde{\phi}}
\newcommand{\Ye}{Y_\varepsilon}
\newcommand{\tw}{\tilde{w}}
\newcommand{\Ae}{A^\varepsilon}
\newcommand{\Be}{B^\varepsilon}
\newcommand{\ue}{u^\varepsilon}
\newcommand{\ie}{\emph{i.e.{}}~}
\newcommand{\st}{\text{~~s.t.~~}}
\newcommand{\norm}[1]{\left|\left|#1\right|\right|}
\newcommand{\sca}[2]{\big<#1, #2\big>}
\newcommand{\cont}[1]{\mathcal{C}^{#1}}
\newcommand{\question}[2]{\paragraph{Question #1.}\textit{#2} \\}
\newcommand{\Hd}{H^1_{\#}}
\newcommand{\td}{\text{d}}
\newcommand{\balpha}{\bm{\alpha}}
\newcommand{\bbeta}{\bm{\beta}}
\newcommand{\blambda}{\bm{\lambda}}
\newcommand{\balphac}{\bm{\alpha}_c}
\newcommand{\bbetac}{\bm{\beta}_c}
\newcommand{\bpsi}{\bm{\psi}}
\newcommand{\xtk}{x_{T,k}}





\title{Rapport TP X01 \\ TP4}
\author{Aurélien Valade}
\date{}

\begin{document}
\maketitle

\section{Introduction}

Dans ce rapport, nous allons nous pencher sur la description et l'implémentation d'un code de résolutions d'équations aux dérivées partielles par méthode
d'éléments finis et résolution hétérogène multi-échelles, termes que nous détaillerons plus loin. Ce travaille se présente en deux parties : tout d'abord
en une analyse du papier de \cite{abdulle2009short}, puis une tentative de ré-implémentation de certaines parties de ce code à l'aide des routines développées
dans les TP précédents. Contrairement au papier qui présente des méthodes pour différents types de maillage en deux ou trois dimensions, nous nous
bornerons à des explications en deux dimensions avec des maillages de simplexes, c'est à dire triangulaires dans notre cas. 

\section{Analyse du papier \& méthodologie du code}

Ce papier a pour but l'implémentation d'un code MATLAB simple et multifonctions à mettre à disposition du publique. Celui ci permettrait la résolution
d'EDP dont la physique amène à considérer des phénomènes physiques couplés sur plusieurs échelles d'espaces distinctes. Pour la résolution de tels
problèmes, une approche frontale serait donc de discrétiser tout l'espace avec un maillage dont les cellules seraient à l'échelle du phénomène physique le
plus petit, mais cela amènerait à d'énormes problèmes numériques très coûteux à résoudre. Parfois la physique demande même à ce que la taille du phénomène
microscopique tende vers zéro, il est alors impossible d'attaquer le problème brutalement et il faut donc créer un meilleur cadre analytique et trouver
des méthodes numériques plus efficaces quitte à faire des approximations.

La littérature semble être fournie en ce qui concerne l'approche analytique de ce genre de problème multi-échelles, cependant peu d'outils numériques
semblent être disponibles aujourd'hui, ce à quoi le travail présenté dans ce papier veut remédier. 

\subsection{Cadre mathématique de la résolution}

\subsubsection{Présentation de l'équation}

L'étude présentée dans ce papier porte sur des méthodes d'éléments finis (FE) pour la résolution de problèmes elliptiques et paraboliques comme suit.

\begin{pb}
  \label{pb:parael}
  Soient les sous espaces $\Omega \subset \R^d$\quad et \quad $Y = [0,1]^d$. Soient $\varepsilon >0$, $\alpha\in \{0, 1\}$ et soient les fonctions sur ces
  espaces $f : \Omega \to \R, \quad \Ae : \Omega\times Y \to \mathcal{M}_d(\R)$  telle que $\Ae  = (\Ae)^T$ \quad et \quad $g_{N,D} : \partial \Omega_{N, D} \to \R$. \\
  Trouver $u$ tel que
  \begin{equation*}
    \begin{cases}
      \alpha \partial_t \ue -\nabla(\Ae \nabla \ue) = f & \text{sur }\Omega\\
      \ue = g_D & \text{sur }\partial \Omega_D \\
      n\cdot(\Ae \nabla \ue) = g_N & \text{sur }\partial \Omega_N
    \end{cases}
  \end{equation*}
\end{pb}

\begin{rmq}
  Il s'agit d'un problème elliptique si $\alpha=0$, sinon il s'agit d'un problème parabolique.
\end{rmq}

\begin{rmq}
  La dépendance en $\varepsilon$ de $\Ae$ n'est pas forcément de la forme vue en cours, \ie périodique $\Ae(x, x/\varepsilon)$, mais elle peut être
  quasi-périodique $\Ae(x, x/\varepsilon)$, ou bien même implicite, bien qu'on ne considère dans la suite que ce cas simple. Il est cependant nécessaire
  d'avoir pour tout $\varepsilon$ les propriétés suivantes :
  \[
    \exists \lambda, \Lambda > 0 \st \lambda \xi^2 < (\Ae \xi) \cdot \xi < \Lambda \xi^2 \quad \forall \xi \in \R^d
  \]
  en tous points pour pouvoir prouver que le problème variationnel vu en \autoref{sec:pbvar} est bien posé.
\end{rmq}

Cette formulation permet de mettre en exergue la dépendance en $\varepsilon$ qui est l'échelle la plus petite du problème, et qui tend usuellement vers
zéro. C'est cette dépendance qui est au centre du questionnement. En effet il ne suffit pas de trouver une solution $\ue$ explicite en $\varepsilon$ et de
faire tendre $\varepsilon$ vers zéro pour trouver $u^0$. La convergence n'est généralement que faible dans $L^2$ :
\[
  \lim_{\varepsilon \to 0} \int_\Omega \ue \phi = \int_\Omega u^0 \phi \qquad \forall \phi \in L^2(\Omega).
\]

Les deux méthodes présentées plus loin permettent de palier à ce problème par des méthodes analytiques amenant à des implémentations numériques modifiées.


\subsubsection{Problème variationnel associé}
\label{sec:pbvar}

On va poser ici la formulation variationnelle discrète du problème pour mieux voir les éléments qui nous intéressent réellement, à savoir les coefficients
de la matrice de rigidité.

Par une démonstration vue plusieurs fois en cours dans un cadre plus restreint\footnote{Multiplication par $\phi$,
  Green-Ostrogradski deux fois dont la deuxième pour changer l'intégrale sur $g_D$.}, on obtient que le problème \ref{pb:parael} est équivalent à
\begin{equation}
  \underbrace{\int_\Omega (\Ae \nabla \ue)\cdot \nabla \phi}_{\Be(\ue, \phi)} =
  \underbrace{\int_\Omega f \phi + \int_{\partial \Omega_N} g_N \phi - \int_{\Omega} (\Ae \nabla g_D)\cdot \nabla \phi}_{l^{\varepsilon}(\phi)}
  \qquad \forall \phi \in L^2(\Omega).
\end{equation}

\begin{rmq}
  On peut réécrire immédiatement
  \[
    l^{\varepsilon}(\phi) = l(\phi) - \Be(g_D, \phi).
  \]
\end{rmq}

Sous certaines conditions de coercitivité et de continuité des fonctions (bi)linéaires en présence, on peut montrer que le problème est bien posé à l'aide
du théorème de Lax-Milgram.

Il apparaît que le point centrale de la difficulté de cette équation est dans la fonction bilinéaire $\Be$. Ce sont en effet ces coefficients que l'on
cherchera à calculer par la méthode la plus adaptée pour quand $\varepsilon$ tend vers 0.

Avant cela, discrétisons cette équation sur une base finie de fonctions de $L^2(\Omega)$. On se donne un maillage de simplexe en deux
dimensions\footnote{Le papier susmentionné va plus loin en considérant d'autres types de maillages, en 2D et en 3D.}. On note les sommets de ce maillage
les $\{S_i\}_{1, N}$ et $\T_h$ l'ensemble des simplexe. On définit l'ensemble des sommets les fonctions de $\R^g[X]$ avec $g\in\N$ le degré des polynômes
choisi. Ces fonctions, notées $\{\varphi_i\}_{1, N}$ par la suite sont telles que
\[
  \varphi_i(S_j) = \delta_{i,j} \quad \forall i, j \in [1, N]^2.
\]
On projette $\ue$ sur cette bases et on utilise les $\{\varphi_i\}_{1, N}$ comme $\phi$ pour obtenir le système linéaire 
\[
  \K^\varepsilon U^\varepsilon = L^\varepsilon
\]
avec $\K^\varepsilon_{ij} = \Be(\varphi_i, \varphi_j), ~ U^\varepsilon_i = \ue(S_i)$ et $L^\varepsilon_i = l^\varepsilon(\varphi_i)$.


\subsection{Méthode d'homogénéisation classique}
  
Le concept développé en homogénéisation pure est de trouver un méthode pour calculer le tenseur $\Ah$ en tout point pour pouvoir ensuite résoudre le
problème dérivé qui nous donne la solution $\uh=\lim_{\varepsilon\to 0}\ue$:
\begin{pb}
  Trouver $\uh$ tel que
  \begin{equation*}
    \alpha \partial_t \uh -\nabla(\Ah \nabla \uh) = f
  \end{equation*}
\end{pb}


L'avantage est ici que $\varepsilon$ a complètement disparu de l'équation et donc qu'un
maillage adapté aux grandes échelles seules suffit. Cependant il faut pour cela calculer $\Ah(\bx)$ au préalable \ie avant la résolution du macro
problème. Pour cela on résout les problèmes dits de cellule en tout point $\bx \in \Omega$ pour trouver les $\{w_i\}_{1, d}$ : 
\begin{equation}
  \label{eq:cell}
  \int_Y A^1(\bx, y) \nabla w_i(\bx, y) \nabla \phi(y) \td y  = \int_Y A^1(\bx, y) e_i \nabla \phi(y) \td y
  \quad  \forall\phi\in L^2(\Omega) \qquad \forall i \in \{1, d\} 
\end{equation}
puis on calcule le tenseur homogénéisé
\begin{equation}
  \label{eq:Ahom}
  \Ah_{jk}(\bx) = \int_Y A(\bx, y)\big(e_k+\nabla_y w_k(\bx, y)\big)\big(e_j+\nabla_y w_j(\bx, y)\big) \td y 
\end{equation}
et on peut finalement résoudre le système linéaire homogénéisé :
\[
  \K U = L
\]
avec $\K_{ij} = \int_\Omega (\Ah \varphi_i)\cdot \varphi_j, ~ U_i = \uh(S_i)$ et $L_i = l(\varphi_i) - \int_\Omega (\Ah g_D)\cdot \varphi_i$.

Le gros inconvénient de cette méthode étant qui faut préalablement calculer le tenseur homogénéisé $\Ah$ et ce théoriquement en tout point de $\Omega$ de
l'espace discrétisé.


\subsection{Différences et apport de la nouvelle méthode}

La méthode de la FE-HMM (\emph{finite elements heterogenous multi-scale method}) présente tous les avantages d'une méthode d'homogénéisation, mais permet
en plus d'éviter le calcul de $\Ah(\bx)$ en un grand nombre des points. On peut ainsi travailler exclusivement avec comme entrée les données initiales,
et calculer le moins de données intermédiaires.

Deux différentes échelles se distinguent :
\begin{itemize}
\item la macro-échelle, de taille caractéristique de maillage $H$
\item et la micro-échelle, de taille caractéristique de maillage $h$,
\end{itemize}
la première étant celle associée du système physique, et la seconde étant une échelle arbitrairement plus petite dont l'usage sera présisé plus loin.

\begin{rmq}
  On parlera dans la suite de macro-maillage (resp. micro-) composé de macro-cellules (resp. micro-) par racourcis de langage pour désigner
  les quantités liées à la macro-échelle (resp. micro-). 
\end{rmq}

On aurait a priori envie d'associer directement la micro-échelle à la taille d'une maille de la macro, \ie de mailler chaque macro-cellule par un
micro-maillage. Cependant la méthode proposée est un peu différente. On va choisir dans macro-cellule un ensemble de points pour faire une quadrature.
Sont proposées deux quadratures :
\begin{itemize}
\item une quadrature avec un unique point au centre de la macro cellule
\item une quadrature de Legendre à deux points en chaque direction de l'espace, on aura donc 4 points en $\{(1/2 \pm \sqrt{3}/6,~1/2 \pm \sqrt{3}/6)\}$
  en considérant une maille centrée en $(0, 0)$,
\end{itemize}
auxquelles on pourrait ajouter d'autres quadratures d'ordre plus élevé par exemple. On va considérer dans la suite que l'on travaille dans le cas de la
quadrature à quatres points.

\begin{rmq}
  Nous allons dans notre code reprendre la quadrature à quatre points proposée dans le TP1 à la place de la quadrature de Gausse-Legendre.
\end{rmq}

Autour de chaque point de cette quadrature on forme un carré de côté $\delta \ll H$. C'est sur ces espaces que l'on va résoudre les problèmes de
cellules, et c'est cette échelle que nous appelont micro-échelle.

\begin{rmq}
  On utilise dans la suite des micro-maillages de côté $\delta = \varepsilon$, ce qui amène d'après le papier à un code robuste et indépendant de
  $\varepsilon$. 
\end{rmq}


\subsection{Modification analytique des problèmes de cellule}

Le problèmes de cellule sur lesquels on travaille sont des problèmes modifiés qui permettent, comme précisé précédemment, de ne pas calculer $\Ah$ mais
directement la matrice de rigidité à partir des données initiales. Les calculs suivant démontrent comment, en partant des équations vues en cours, on
peut retrouver les problèmes de cellule modifiés qui seront résolus dans le code.

\begin{rmq}
  Ces problèmes sont au nombre de sommets par cellule. Comme on utilise uniquement des simplexes de dimention deux, on aura donc trois problèmes
  modifiés par macro-cellule. Pour faire l'analogie avec les codes vus dans les TP précédents, ces problèmes permettent en réalité de contruire
  directement les matrices élémentaires \texttt{Kel} de la matrice de rigidité.
\end{rmq}

On veut montre d'abord que l'\autoref{eq:Ahom} peut se réécrire sous la forme de l'\autoref{eq:AhomYeps}:

\begin{equation}
  \label{eq:AhomYeps}
  \Ah_{jk}(\bx) = \frac{1}{|\Ye|}
  \int_{\Ye} A(\bx, x/\varepsilon) \big(e_k+\nabla_x \eta_k(\bx, x/\varepsilon)\big)\big(e_j+\nabla_x \eta_j(\bx, x/\varepsilon)\big) dx
\end{equation}
avec 
\[
  \eta_i = \varepsilon w_i \qquad \forall i \in \{1,2\}
\]

On procède par changement de variable : $ y \rightarrow x/\varepsilon $ et en faisant les transformations suivantes
\begin{itemize}
\item $dy \rightarrow dx/\varepsilon = dx/|\Ye|$,
\item $y=(-1/2, -1/2) \rightarrow x=(-\varepsilon/2, -\varepsilon/2) \implies Y \rightarrow \Ye = (-\varepsilon/2, -\varepsilon/2)^2$,
\item $\nabla_y = \frac{\partial}{\partial y} = \frac{\partial x}{\partial y}\frac{\partial}{\partial x} = \varepsilon \nabla_x $,
\end{itemize}
la réécriture est immédiate.

On veut ensuite montrer que le problème de cellule est ``translatable'', c'est à dire que l'on peut travailler autour d'un point quelconque
$\bx$. Cela est direct si on remarque quela fonction étant périodique et l'interval d'integration étant précisément un interval de périodicité de la
fonction, la translation de ce dernier n'influe pas sur l'intégrale :
\begin{equation}
  \label{eq:pbcelletat}
  \int_{\Ye(\bx)} A(\bx, x/\varepsilon)\nabla_x \eta_i(\bx, x/\varepsilon) \nabla_x \phi(x) dx =
  - \int_{\Ye(\bx)} A(\bx, x/\varepsilon) e_i \nabla_x \phi(x) dx
\end{equation}

On retrouve alors l'équation sur $Y$ avec les $w_i$ des problèmes de cellule :
\[
    \int_{Y} A(y)\nabla_y w_i(y) \nabla_y \phi(\bx+\varepsilon y) dy =
  - \int_{Y} A(y) e_i \nabla_y \phi(\bx+\varepsilon y) dy \qquad \forall i \in \{1, 2\}, \forall \phi\in V_\varepsilon(\Ye)
\]
mais on voudrait montrer cette égalité $\forall \tphi \in V$, ce qui est équivalent car

\[
  \forall \tphi \in V, \exists \phi \in V_\varepsilon(\Ye) \quad \text{tq} \quad \tphi(x) =
  \phi(\bx+\varepsilon y) \text{ en tout points associés } x \rightarrow y
\]
or l'\autoref{eq:pbcelletat} étant vraie $\forall \phi \in V_\varepsilon(\Ye)$, on peut bien écrire 

\begin{equation}
  \int_{Y} A(y)\nabla_y w_i(y) \nabla_y \phi(y) dy =
  - \int_{Y} A(y) e_i \nabla_y \phi(y) dy \qquad \forall i \in \{1, 2\}, \forall \phi\in V.
\end{equation}

Le but de la méthode étant d'éviter le calcul intermédiaire de $\Ah$, on va écrire analytiquement les coefficients de $\K$ qui nous
intéressent. Ainsi, es prochaines équations construisent la contribution d'un simplexe à un coefficients $\K_{IJ}$ représentant le recouvrement entre
deux éléments de la base globale appartenant noté dans la base locale $w_i$ et $w_j$, $i,j\in [1,3]$. Pour alléger le calcul, on pose les conventions
d'écriture suivantes
\begin{itemize}
\item
  $ 
  \Ah = \left(
    \begin{matrix}
      \Ah_{11} & \Ah_{12} \\
      \Ah_{21} & \Ah_{22}
    \end{matrix}
  \right)
  $
\item $ \partial_{x_i} \rightarrow \partial_i, \quad \forall i \in\{1,2\}$  
\item 
  $
  B = \left(
    \begin{matrix}
      1+\partial_1 \eta_1 &   \partial_1 \eta_2 \\
        \partial_2 \eta_1 & 1+\partial_2 \eta_2
    \end{matrix}
  \right)
  = \left(
    \begin{matrix}
      e_1 + \nabla_x \eta_1 & e_2 + \nabla_x \eta_2
    \end{matrix}
  \right)
  = \left(
    \begin{matrix}
      B_1 & B_2
    \end{matrix}
  \right)
  = I + \nabla_x \eta
  $
\end{itemize}
et on ne note pas les arguments des fonctions. Commençons par écrire $\Ah\nabla_x w_j$ :
\begin{equation}
  \begin{aligned}
    \Ah\nabla_x w_j &=
    \left(
      \begin{matrix}
        \Ah_{11} & \Ah_{12} \\
        \Ah_{21} & \Ah_{22}
      \end{matrix}
    \right)
    \left(
      \begin{matrix}
        \partial_1 w_j \\
        \partial_2 w_j
      \end{matrix}
    \right) \\
    &=
    \left(
      \begin{matrix}
        \Ah_{11} \partial_1 w_i + \Ah_{12} \partial_2 w_i \\
        \Ah_{21} \partial_1 w_i + \Ah_{22} \partial_2 w_i 
      \end{matrix}
    \right)
  \end{aligned}
\end{equation}
on peut maintenant multiplier par la droite
\begin{equation}
  \begin{aligned}
    \Ah\nabla_x w_j \nabla_x w_i =(\Ah_{11} \partial_1& w_i + \Ah_{12} \partial_2 w_i) \partial_1 w_i
    + (\Ah_{21} \partial_1 w_i + \Ah_{22} \partial_2 w_i ) \partial_2 w_i \\
    = \frac{1}{|\Ye|} \int_{\Ye} &A(e_1 + \nabla_x \eta_1)\cdot(e_1 + \nabla_x \eta_1) \partial_1 w_j \partial_1 w_i + \\
    &A(e_2 + \nabla_x \eta_2)\cdot(e_1 + \nabla_x \eta_1) \partial_2 w_j \partial_1 w_i +\\ 
    &A(e_1 + \nabla_x \eta_1)\cdot(e_2 + \nabla_x \eta_2) \partial_1 w_j \partial_2 w_i +\\
    &A(e_2 + \nabla_x \eta_2)\cdot(e_2 + \nabla_x \eta_2) \partial_2 w_j \partial_2 w_i  \\
    = \frac{1}{|\Ye|} \int_{\Ye} &A B_1 \partial_1 w_j \cdot B_1 \partial_1 w_i + A B_2 \partial_2 w_j \cdot B_1 \partial_1 w_i +\\ 
    &A B_1 \partial_1 w_j \cdot B_2 \partial_2 w_i + A B_2 \partial_2 w_j \cdot B_2 \partial_2 w_i \big] \\
    = \frac{1}{|\Ye|} \int_{\Ye} &A (B_1 \partial_1 w_j + B_2 \partial_2 w_j ) \cdot B_1 \partial_1 w_i + \\
    &A (B_1 \partial_1 w_j  + B_2 \partial_2 w_j )\cdot B_2 \partial_2 w_i +\\
    = \frac{1}{|\Ye|} \int_{\Ye} &A  B\nabla_x w_j \cdot B\nabla_x w_i  \\
    = \frac{1}{|\Ye|} \int_{\Ye} &A(I + \nabla_x \eta) \nabla_x w_j (I + \nabla_x \eta) \nabla_x w_i
  \end{aligned}
\end{equation}
cette équation appliquée aux points corrects donne $\forall i,j\in [1,3]$
\begin{equation*}
  \begin{aligned}
    \Ah\nabla_x w_j \nabla_x w_i = \frac{1}{|\Ye(x_{T, k})|} \int_{\Ye(x_{T, k})} A(x_{T, k}, x/\varepsilon)&(I + \nabla_x \eta(x_{T, k},
    x/\varepsilon)) \nabla_x w_j(x_{T, k}) \\
    &(I + \nabla_x \eta(x_{T, k}, x/\varepsilon)) \nabla_x w_i(x_{T, k})
  \end{aligned}
\end{equation*}

On veut enfin écrire le problème modifié dans la forme définitive ou apparaît le couplage en fonctions de la micro-base et celles de la macro-base.
En passant le membre de droite dans l'\autoref{eq:pbcelletat}, on retrouve
\begin{equation}
  \begin{aligned}
    &\int_{\Ye(\xtk)} A(\xtk, x/\varepsilon)(e_i + \nabla_x \eta_i(\xtk, x/\varepsilon)) \nabla_x \phi(x) dx = 0 \\
    \iff &\int_{\Ye(\xtk)} A(\xtk, x/\varepsilon)B_i \nabla_x \phi(x) dx = 0 \\
  \end{aligned}
\end{equation}
on peut donc écrire les couples d'équations scalaires $\forall i\in \{1,2\}$, avec $l\in [1, 3]$.
\begin{equation}
  \begin{aligned}
    &\begin{cases}
      \int_{\Ye(\xtk)} A(\xtk, x/\varepsilon) B_1(\xtk, x/\varepsilon) \nabla_x \phi(x) (\nabla w_l)_1 dx = 0 \qquad i = 1 \\
      \int_{\Ye(\xtk)} A(\xtk, x/\varepsilon) B_2(\xtk, x/\varepsilon) \nabla_x \phi(x) (\nabla w_l)_2 dx = 0 \qquad i = 2 \\ 
    \end{cases}  \\                     
    \iff &\begin{cases}                          
      \int_{\Ye(\xtk)} A(\xtk, x/\varepsilon) B_1(\xtk, x/\varepsilon) (\nabla w_l)_1 \nabla_x \phi(x) dx = 0 \qquad i = 1 \\
      \int_{\Ye(\xtk)} A(\xtk, x/\varepsilon) B_2(\xtk, x/\varepsilon) (\nabla w_l)_2 \nabla_x \phi(x) dx = 0 \qquad i = 2 \\ 
    \end{cases} 
  \end{aligned}
\end{equation}

d'où 
\begin{equation}
  \label{eq:pbcelltw}
  \int_{\Ye(\xtk)} A(\xtk, x/\varepsilon)\nabla_x \tw_l(\xtk, x/\varepsilon) \nabla_x \phi(x) dx = 0 \quad \forall l \in [1, 3]\\
\end{equation}
avec
\[
  \nabla_x \tw_l(\xtk, x/\varepsilon) = B(\xtk, x/\varepsilon) \nabla_x w_l(\xtk).
\]

Il est important de noter qu'il ne s'agit plus de résoudre un problème hyperbollique comme jusqu'à présent, puisqu'une contrainte supplémentaire est
imposée :
\[
  \tw_l \in w_{l, lin} + \Hd(\Ye).
\]
\begin{rmq}
  Cette contrainte est importante, sans elle le problème serait mal posé, la fonction nulle étant une alternative à celle recherchée.
\end{rmq}

\subsection{Résolution numérique des nouveaux problèmes de cellule}

\subsubsection{Système contraint}


En suivant les récommandation du papier, on va donc changer le système à résoudre en passant par une méthode lagrangienne. Cette nouvelle formulation
devra donc prendre en compte les deux contraintes en présence :
\begin{itemize}
\item La périodicité des problèmes de cellule modifiés
\item La contrainte sur l'espace de définition de la solution.
\end{itemize}
Discrétisons nos espaces,  soient
\begin{itemize}
\item $\{\psi_i\}_{M_{mic}}$ la micro-base \ie la base de $\Ye(\xtk)$ 
\item  $\{\varphi_i\}_{1,2,3}$ la macro-base \textbf{locale}.
\end{itemize}
On projette sur la micro-base :
\begin{itemize}
\item $\tw_l = \sum \alpha^i_l \psi_i = \balpha_l \cdot \bpsi$ \quad les solutions aux problèmes de cellule modifiés,
\item $\varphi_l = \sum \beta^i_l \psi_i = \bbeta_l \cdot \bpsi$ \quad les éléments de la base locale.
\end{itemize}
Les inconnues sont maintenant les vecteurs $\balpha_l$, le reste étant connu ou aisément calculable.
\begin{rmq}
  On rappelle qu'il y a donc trois vecteurs inconnus par points de quadrature, bien qu'on ne surcharge pas la notation avec un indice $k$ sur les
  $\balpha_l$.
\end{rmq}
On va poser la matrice $\D$ qui
regroupera nos contraintes, et on se donne la matrice de rigidité locale $\Ktk$ calculée de manière classique sur l'intervale $\Ye(\xtk)$. On va
chercher à résoudre les systèmes
\begin{equation}
  \label{eq:sysconst}
  \begin{cases}
    \Ktk \balpha_l + \D \blambda_l = 0 \\
    \D^t(\balpha_l - \bbeta_l) = 0 
  \end{cases} \qquad \forall l \in [1, 3]
\end{equation}
où $\blambda_l \in \R^C$ est un multiplicateur de lagrange et $C$ le nombre de contraintes. On va réécrire ces systèmes en un seul où les
solutions sont concaténées
\begin{equation}
  \Ktkc \balphac = \bbetac
\end{equation}
avec
\[
  \Ktkc = \left(
    \begin{tabular}{ccc|c}
      ~ & ~     & ~ & ~      \\
      ~ & $\Ktk$ & ~ & $\D$ \\
      ~ & ~     & ~ & ~      \\
      \hline
      ~ & $\D^t$  & ~ & 0
    \end{tabular}
  \right)
  ~,\qquad
  \balphac = \left(
    \begin{matrix}
      ~         & ~         & ~         \\
      \balpha_1 & \balpha_2 & \balpha_3 \\
      ~         & ~         & ~         \\
      \blambda_1 & \blambda_2 & \blambda_3
    \end{matrix}
  \right)
  ~,\qquad
    \bbetac = \left(
    \begin{matrix}
      ~          & ~          & ~ \\
      0          & 0          & 0 \\
      ~          & ~          & ~ \\
      \D^t\bbeta_1 & \D^t\bbeta_2 & \D^t\bbeta_3  
    \end{matrix}
  \right).
\]

Cela revient bien sûr à résoudre les trois systèmes séparément, mais la concaténation est utile dans la suite et simplifie l'écriture. En effet, une
fois ce système résolu, on peut directement calculer la contribution de la cellule $k$ à la matrice de rigidité globale
\[
  \Kk = \sum_{\xtk} \frac{\omega_{\xtk}}{\varepsilon} (\J\balphac)^t \Ktk (\J\balphac)
  \quad \text{avec} \quad
  \J = \left(
    \begin{matrix}
      I_{M_{mic}} & 0 \\
      0           & 0
    \end{matrix}
  \right).
\]


\subsubsection{Matrice de contrainte}

La matrice de contrainte est le point dur de la construction du système contraint. Elle contient les deux contraintes
\begin{enumerate}
\item $\int_{\Ye(\xtk)} \tw_l - w_{l, lin} = 0$
\item $(\tw_l - w_{l, lin})(p) = (\tw_l - w_{l, lin})(p')$ pour tout couple $(p, p')$ lié par les conditions périodiques \textbf{sans} redondance.
\end{enumerate}
Cette matrice sera donc la forme
\[
  \D = \left(
    \begin{matrix}
      b_1 & \dots & b_{M_{mic}} \\
      ~ &  \tD & ~\\
    \end{matrix}
  \right)
\]
où $\tD \in \mathcal{M}_{(M_{mic}, \smbox{Nbaretes})}(\R)$ est la matrice qui force les conditions periodiques. Elle est de la forme suivante pour
notre numérotation des noeuds :
\[
  \tD^t = \left(
    \begin{tabular}{p{.2\textwidth}|p{.2\textwidth}|p{.2\textwidth}|p{.2\textwidth}}
      Liens entre angles
      $(1)\leftrightarrow(2)$,
      $(3)\leftrightarrow(4)$,
      $(1)\leftrightarrow(4)$
      &
        Liens horizotaux sur les côtés droite et gauche
      &
        Liens verticaux sur les côtés droite et gauche
      &
        0 pour les noeuds du centre
    \end{tabular}
  \right)
\]
On peut montrer que pour la containte (1), on peut prendre 
\[
  b_i = \int_{\Ye} \psi_i.
\]

\section{Implémentation du code}

Nous allons maintenant développer quelques éléments important à la compréhension du code. 
\end{document}
%%% Local Variables:
%%% mode: latex
%%% TeX-master: t
%%% End:


% Après second changement de variable $x\rightarrow x +\bx$ on travaille avec les nouvelles fonctions $\eta_i = \varepsilon w_i(\bx, (x-\bx)/\varepsilon)$.
% On voudrait montrer que ces nouvelles fonctions résolvent le problème \autoref{eq:pbcelleta}

% pour cela, on peut résonner par équivalence en faisant le chemin inverse en faisant les changements de variables inverses
% \begin{enumerate}
% \item $x\rightarrow x+\bx$
% \item $x\rightarrow \varepsilon x$
% \end{enumerate}
% \begin{rmq}
%   Par des arguments de pseudo périodicité on peut montrer que le premier changement de variable ne change pas $A(x, x/\varepsilon)$.
% \end{rmq}

avec une disjonction de cas on trouve
\begin{equation}
  \begin{aligned}
    &\begin{cases}
      \int_{\Ye(\xtk)} A(\xtk, x/\varepsilon) B_1(\xtk, x/\varepsilon) (\nabla w_1)_1 \nabla_x \phi(x) dx = 0 \\
      \int_{\Ye(\xtk)} A(\xtk, x/\varepsilon) B_2(\xtk, x/\varepsilon) (\nabla w_1)_2 \nabla_x \phi(x) dx = 0 \\
    \end{cases} && l=1\\
    \text{et}~~
    &\begin{cases}
      \int_{\Ye(\xtk)} A(\xtk, x/\varepsilon)B_1(\xtk, x/\varepsilon) (\nabla w_2)_1 \nabla_x \phi(x) dx = 0 \\
      \int_{\Ye(\xtk)} A(\xtk, x/\varepsilon)B_2(\xtk, x/\varepsilon) (\nabla w_2)_2 \nabla_x \phi(x) dx = 0 \\ 
    \end{cases} && l=2\\
    \iff
    &\begin{cases}
      \int_{\Ye(\xtk)} A(\xtk, x/\varepsilon)B(\xtk, x/\varepsilon) \nabla w_1 \nabla_x \phi(x) dx = 0 \\
      \int_{\Ye(\xtk)} A(\xtk, x/\varepsilon)B(\xtk, x/\varepsilon) \nabla w_2 \nabla_x \phi(x) dx = 0 \\ 
    \end{cases} \\
    \iff
    &\begin{cases}
      \int_{\Ye(\xtk)} A(\xtk, x/\varepsilon)\nabla_x \tw_1(\xtk, x/\varepsilon) \nabla_x \phi(x) dx = 0 \\
      \int_{\Ye(\xtk)} A(\xtk, x/\varepsilon)\nabla_x \tw_2(\xtk, x/\varepsilon) \nabla_x \phi(x) dx = 0 \\ 
    \end{cases}
  \end{aligned}
\end{equation}