\message{ !name(rapport.tex)}\documentclass[11pt]{article}
\usepackage[utf8]{inputenc}

% MATH
\usepackage{amssymb}
\usepackage{amsmath}
\usepackage{amsthm}
\usepackage{mathtools}

%\newtheorem{cusdef}{Def}
%\newtheorem{custhm}{Thm}
%\newtheorem{cuscor}{Cor}
%\newtheorem{cusrk}{Rk}
\newenvironment{cusenv}[2]
{\begin{samepage}\noindent\textbf{#1} -- (#2) \par}
{\end{samepage} \bigskip}

\newenvironment{cusdef}[1]
{\begin{cusenv}{Def}{#1}}{\end{cusenv}}

\newenvironment{custhm}[1]
{\begin{cusenv}{Thm}{#1}}{\end{cusenv}}

\newenvironment{cuscor}[1]
{\begin{cusenv}{Cor}{#1}}{\end{cusenv}}

\newenvironment{cusprop}[1]
{\begin{cusenv}{Prop}{#1}}{\end{cusenv}}

\newenvironment{cusrk}[1]
{\begin{cusenv}{Rk}{#1}}{\end{cusenv}}	

% MISE EN FORME
\usepackage[left=2cm,right=2cm,top=2cm,bottom=2cm]{geometry}

\usepackage{hyperref}
\usepackage{xcolor}
\hypersetup{
	colorlinks,
	linkcolor={red!50!black},
	citecolor={blue!50!black},
	urlcolor={blue!80!black}
}


% MACROS
\usepackage{xparse}
\newcommand{\smbox}[1]{\mbox{\footnotesize #1}}
\newcommand{\Am}{\mathbb{A}}
\newcommand{\N}{\mathbb{N}}
\newcommand{\C}{\mathbb{C}}
\renewcommand{\P}{\mathbb{P}}
\newcommand{\R}{\mathbb{R}}
\newcommand{\K}{\mathbb{K}}
\newcommand{\M}{\mathbb{M}}
\newcommand{\Ah}{A^{hom}}
\newcommand{\bx}{\bar{x}}
\newcommand{\teta}{\tilde{\eta}}
\newcommand{\tphi}{\tilde{\phi}}
\newcommand{\Ye}{Y_\varepsilon}
\newcommand{\tw}{\tilde{w}}
\newcommand{\ie}{\emph{i.e.} }
\newcommand{\norm}[1]{\left|\left|#1\right|\right|}
\newcommand{\sca}[2]{\big<#1, #2\big>}
\newcommand{\cont}[1]{\mathcal{C}^{#1}}
\newcommand{\question}[2]{\paragraph{Question #1.}\textit{#2} \\}
\newcommand{\Hd}{H^1_{\#}}



\title{Rapport TP X01 \\ TP4}
\author{Aurélien Valade}
\date{}

\begin{document}

\message{ !name(rapport.tex) !offset(155) }
  \begin{cases}
    \int_{\Ye(x_{T,k})} A(x_{T,k}, x/\varepsilon) B_i(x_{T,k}, x/\varepsilon) \nabla_x \phi(x) (\nabla w_l)_1 dx = 0 \\
    \int_{\Ye(x_{T,k})} A(x_{T,k}, x/\varepsilon) B_i(x_{T,k}, x/\varepsilon) \nabla_x \phi(x) (\nabla w_l)_2 dx = 0 \\ 
  \end{cases} \iff                        
  \begin{cases}                          
    \int_{\Ye(x_{T,k})} A(x_{T,k}, x/\varepsilon) B_i(x_{T,k}, x/\varepsilon) (\nabla w_l)_1 \nabla_x \phi(x) dx = 0 \\
    \int_{\Ye(x_{T,k})} A(x_{T,k}, x/\varepsilon) B_i(x_{T,k}, x/\varepsilon) (\nabla w_l)_2 \nabla_x \phi(x) dx = 0 \\ 
  \end{cases}
\end{equation}
donc avec une disjection de cas on trouve
\begin{equation}
  \begin{aligned}
    &\begin{cases}
      \int_{\Ye(x_{T,k})} A(x_{T,k}, x/\varepsilon) B_1(x_{T,k}, x/\varepsilon) (\nabla w_1)_1 \nabla_x \phi(x) dx = 0 \\
      \int_{\Ye(x_{T,k})} A(x_{T,k}, x/\varepsilon) B_2(x_{T,k}, x/\varepsilon) (\nabla w_1)_2 \nabla_x \phi(x) dx = 0 \\
    \end{cases}
    \quad et \quad
    \begin{cases}
      \int_{\Ye(x_{T,k})} A(x_{T,k}, x/\varepsilon)B_1(x_{T,k}, x/\varepsilon) (\nabla w_2)_1 \nabla_x \phi(x) dx = 0 \\
      \int_{\Ye(x_{T,k})} A(x_{T,k}, x/\varepsilon)B_2(x_{T,k}, x/\varepsilon) (\nabla w_2)_2 \nabla_x \phi(x) dx = 0 \\ 
    \end{cases} \\
    &\begin{cases}
      \int_{\Ye(x_{T,k})} A(x_{T,k}, x/\varepsilon)B(x_{T,k}, x/\varepsilon) \nabla w_1 \nabla_x \phi(x) dx = 0 \\
      \int_{\Ye(x_{T,k})} A(x_{T,k}, x/\varepsilon)B(x_{T,k}, x/\varepsilon) \nabla w_2 \nabla_x \phi(x) dx = 0 \\ 
    \end{cases}
    \iff
    \begin{cases}
      \int_{\Ye(x_{T,k})} A(x_{T,k}, x/\varepsilon)\nabla_x \tw_1(x_{T,k}, x/\varepsilon) \nabla_x \phi(x) dx = 0 \\
      \int_{\Ye(x_{T,k})} A(x_{T,k}, x/\varepsilon)\nabla_x \tw_2(x_{T,k}, x/\varepsilon) \nabla_x \phi(x) dx = 0 \\ 
    \end{cases}
  \end{aligned}
\end{equation}
d'ou $\forall l \in \{1,2\}$
\begin{equation}
  \label{eq:pbcelltw}
  \int_{\Ye(x_{T,k})} A(x_{T,k}, x/\varepsilon)\nabla_x \tw_l(x_{T,k}, x/\varepsilon) \nabla_x \phi(x) dx = 0 \\
\end{equation}                                       

\message{ !name(rapport.tex) !offset(152) }

\end{document}
%%% Local Variables:
%%% mode: latex
%%% TeX-master: t
%%% End:
